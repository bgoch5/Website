\documentclass[]{article}
\usepackage{lmodern}
\usepackage{amssymb,amsmath}
\usepackage{ifxetex,ifluatex}
\usepackage{fixltx2e} % provides \textsubscript
\ifnum 0\ifxetex 1\fi\ifluatex 1\fi=0 % if pdftex
  \usepackage[T1]{fontenc}
  \usepackage[utf8]{inputenc}
\else % if luatex or xelatex
  \ifxetex
    \usepackage{mathspec}
  \else
    \usepackage{fontspec}
  \fi
  \defaultfontfeatures{Ligatures=TeX,Scale=MatchLowercase}
\fi
% use upquote if available, for straight quotes in verbatim environments
\IfFileExists{upquote.sty}{\usepackage{upquote}}{}
% use microtype if available
\IfFileExists{microtype.sty}{%
\usepackage{microtype}
\UseMicrotypeSet[protrusion]{basicmath} % disable protrusion for tt fonts
}{}
\usepackage[margin=1in]{geometry}
\usepackage{hyperref}
\hypersetup{unicode=true,
            pdftitle={Publications},
            pdfborder={0 0 0},
            breaklinks=true}
\urlstyle{same}  % don't use monospace font for urls
\usepackage{graphicx,grffile}
\makeatletter
\def\maxwidth{\ifdim\Gin@nat@width>\linewidth\linewidth\else\Gin@nat@width\fi}
\def\maxheight{\ifdim\Gin@nat@height>\textheight\textheight\else\Gin@nat@height\fi}
\makeatother
% Scale images if necessary, so that they will not overflow the page
% margins by default, and it is still possible to overwrite the defaults
% using explicit options in \includegraphics[width, height, ...]{}
\setkeys{Gin}{width=\maxwidth,height=\maxheight,keepaspectratio}
\IfFileExists{parskip.sty}{%
\usepackage{parskip}
}{% else
\setlength{\parindent}{0pt}
\setlength{\parskip}{6pt plus 2pt minus 1pt}
}
\setlength{\emergencystretch}{3em}  % prevent overfull lines
\providecommand{\tightlist}{%
  \setlength{\itemsep}{0pt}\setlength{\parskip}{0pt}}
\setcounter{secnumdepth}{0}
% Redefines (sub)paragraphs to behave more like sections
\ifx\paragraph\undefined\else
\let\oldparagraph\paragraph
\renewcommand{\paragraph}[1]{\oldparagraph{#1}\mbox{}}
\fi
\ifx\subparagraph\undefined\else
\let\oldsubparagraph\subparagraph
\renewcommand{\subparagraph}[1]{\oldsubparagraph{#1}\mbox{}}
\fi

%%% Use protect on footnotes to avoid problems with footnotes in titles
\let\rmarkdownfootnote\footnote%
\def\footnote{\protect\rmarkdownfootnote}

%%% Change title format to be more compact
\usepackage{titling}

% Create subtitle command for use in maketitle
\providecommand{\subtitle}[1]{
  \posttitle{
    \begin{center}\large#1\end{center}
    }
}

\setlength{\droptitle}{-2em}

  \title{Publications}
    \pretitle{\vspace{\droptitle}\centering\huge}
  \posttitle{\par}
    \author{}
    \preauthor{}\postauthor{}
    \date{}
    \predate{}\postdate{}
  

\begin{document}
\maketitle

\hypertarget{section}{%
\section{\texorpdfstring{\textbf{2020}}{2020}}\label{section}}

\textbf{Southwestern Naturalist (Under Review)}

\emph{Filling the gap: molting behavior of Colima Warblers and research
opportunities for understudied North American songbirds}

Benjamin Gochanour, Jose L. Alcantara, Paula Cimprich, Jeffrey F. Kelly,
Andrea Contina

Abstract: We implemented stable isotope analysis to evaluate the molt of
the Colima Warbler (Leiothlypis crissalis), an understudied migratory
songbird occurring in Mexico and recently discovered breeding in the
southern part of Texas, USA. We built a geostatistical model showing
variation in deuterium precipitation values (δ²Hp) across a latitudinal
gradient within the Colima Warbler breeding range in northeastern Mexico
. Then, based on stable isotope ratios of deuterium in feathers (δ²Hf),
we assigned wintering Colima Warblers captured in Central Mexico to
possible molting areas near the southwestern portion of the recognized
species breeding range. To the best of our knowledge, we provide the
first records of the species occurring within the Parque Ecológico de la
Ciudad de México, near the mountain ranges surrounding the Basin of
Mexico. Overall, our study demonstrates the potential of winter ecology
field work in conjunction with molecular study techniques, such as
stable isotope analysis, for revealing the migratory and molting
behavior of warblers with restricted distribution ranges.

\includegraphics[width=0.7\textwidth,height=0.7\textheight]{./tex2pdf.-3de7d20a9a981e0d/46289c16149e31041f15260911580074b7d8f25d.jpg}

\textbf{Biometrics (Submission in Progress)}

\emph{A Nonparametric Multiply Robust Multiple Imputation Method for
Causal Inference}

Benjamin Gochanour, Sixia Chen, Laura Beebe, David Haziza

Abstract: Evaluating the impact of non-randomized treatment on various
health outcomes is difficult in observational studies because of the
presence of covariates that may affect both the treatment or exposure
received and the outcome of interest. In the present study, we develop a
nonparametric multiply robust multiple imputation method for estimating
average treatment effects in such studies, approaching the challenge
from the perspective of potential outcomes. Our method relies on
multiple propensity score models and outcome regression models and is
multiply robust in that it performs well if at least one of the models
is correctly specified. We develop the asymptotic properties of our
method and test it in a simulation study, evaluating its performance in
terms of bias, efficiency, and coverage probability. Rubin's variance
estimation formula can be used safely for estimating the variance of our
proposed estimators. Finally, we apply our method to NHANES data to
examine the effect of exposure to perfluoroalkyl acids (PFAs) on kidney
function.


\end{document}
